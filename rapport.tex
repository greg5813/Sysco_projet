\documentclass[a4paper,12pt]{article}
\usepackage[utf8]{inputenc}
\usepackage[T1]{fontenc}
\usepackage[francais]{babel}
\usepackage{graphicx}
\usepackage[left=3cm,right=3cm,top=4cm,bottom=4cm]{geometry}
\pagestyle{plain}

\title{Projet de Systèmes Concurrents}
\author{Jorge Gutierrez \& Grégoire Martini}
\date{25 Janvier 2016}

\begin{document}
\maketitle
\tableofcontents
\newpage


\section{Etape 1}


\section{Etape 2}


\section{Etape 3}

On doit  implanter un générateur de stub, destiné à soulager le programmeur de l'utilisation explicite des  SharedObject. 
Nous prenons les hypothèses suivantes:

un  objet  partagé  est  une  classe sérialisable  (par  exemple  Sentence).  Il  s'agit  de  la  classe  métier  de  l'objet partagé.

l'objet  partagé  est  utilisable  à  partir  de  variables  de  type  une interface  (par  convention,  l'interface  pour utiliser  un  objet  partagé  de  classe  Sentence  s'appellera  Sentenceitf)  qui  inclut  les  méthodes  métier  de 
Sentence  auquel  on  ajoute  les  méthodes  de  verrouillage  en  héritant  de  SharedObjectitf.  Mais  Sentence n'implémente pas Sentenceitf, car la classe métier ne définit pas les méthodes de verrouillage (c'est le stub qui le fait).

un  stub  est  généré,  appelé  Sentencestub.  Ce  stub  hérite  de  SharedObject  (donc  des  méthodes  de verrouillage) et  il implémente l'interface Sentenceitf.
Avec ces hypothèses, les interfaces de la classe Client (notamment pour le serveur de nom) restent les mêmes. 
Et pour utiliser un objet partagé de la classe Sentence, on fera :
SentenceItf s = (SentenceItf)Client.lookup ("MySentence");
s.lockread();
s.meth();
s.unlock();
On pourra également étudier l'annotation des méthodes dans les interfaces afin de leur associer un mode de verrouillage (@read ou @write). Ainsi, le verrouillage de l'objet partagé est réalisé par le stub généré et on obtient la 
séquence d'appel suivante
:
SentenceItf s = (SentenceItf)Client.lookup ("MySentence");
s.meth();

\section{Etape 4}



\end{document}          
